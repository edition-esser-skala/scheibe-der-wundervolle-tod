\documentclass{ees}

\begin{document}

\eesTitlePage

\eesCriticalReport{
  – & –  & A       & S 2 may be replaced by A throughout the piece \\
}

Maria (A = Canto II)
Eine aus den Weibern (S/A)
Ein andrer aus dem Volke (T)
Einer aus dem Volke (B)
Der römiſche Hauptmann (B)


\eesToc{
  \begin{movement}{wosindwir}
    \voice[ Das Volk]
    Wo ſind wir? wir? die Furcht und Schrecken
    und grauſe Dunkelheit bedecken?
    Wir fühlen des Todes erſchreckliche Nacht!
    O! deine Rache, Jehovah! erwacht!
    Du zürneſt — du rächeſt — wen? dieſen Verbrecher? —
    Nein, Herr! du biſt nicht des Gekreuzigten Rächer.
    Du ſtrafeſt? — wer ſündigte? — tödtende Nacht!
    Frolocke, Mizraim! dein Schickſal erwacht!
    Der Herr hat ſein Juda dir ähnlich gemacht!
  \end{movement}

  \begin{movement}{werduauch}
    \voice[Einer aus dem Volke]
    Wer du auch ſeyſt, den man dort angeheftet hat,
    wer du auch ſeyſt, Mesſias, oder ein Verräther,
    Gott, oder auch ein Miſſethäter!
    Du biſt der Mann, der nun, o ſchwarze That!
    die grauſe Finſterniß erſchaffen.
    Du tödteſt uns mit neuerfundnen Waffen!
    O! richteten die Prieſter auch wohl recht? —

    \voice[Ein andrer aus dem Volke]
    Unglücklichs Land! unſeliges Geſchlecht!
    Verloren ſind wir ſchon. Es nähert ſich die Stunde
    des Unglücks, des Verderbens,
    des Todes, des niemals erhörten Sterbens.
    Verdammter Rath! durch dich geht Gottes Volk zu Grunde.
    Ohnfehlbar iſt er der Mesſias, und unſchuldig.
    Hoch rächet er ſich nun. —

    \voice[Die Weiber und die Gläubigen]
    \hspace*{3cm}Ja! Heil! Mesſias! Gott!
    Unſchuldig biſt du, Herr! Du, deiner Feinde Spott!

    \voice[Eine aus den Weibern]
    Ja! dies bezeugt die nun erblaßte Sonne.
    Du ſtirbſt. Es ſtirbt mit dir auch unſre Wonne
    und unſre Hoffnung —

    \voice[Die Weiber und die Gläubigen]
    \hspace*{3cm}Herr! nun rächet dich der Himmel!

    \voice[Einer aus dem Volke]
    Verdammtes Urtheil! Pilatus! Prieſter! Rath!

    \voice[Das Volk]
    Man tödte ſie, die Schuldgen dieſer That!

    \voice[Der römiſche Hauptmann]
    Welch raſendes Getümmel!
    Volk, voller Angſt! welch eine finſtre Wut!
    Wie? foderſt du noch deiner Vorgeſetzten Blut?
    Soldaten! wehrt dem Aufruhr! —

    \voice[Einer aus dem Volke]
    \hspace*{3cm}Halt!
    Nicht du, nicht deine Schaar hemmt Eifer und Gewalt;
    dies thut ein ſtärkrer Arm, ein höhrer Ruf.
    Die Finſterniß, durch die kein Strahl des Lichtes dringet,
    hält uns zurück. Die iſts, die unſer Wüten zwinget.
    Doch nein! vielmehr der, der ſie ſchuf.
    Und, o Gewaltiger! Jehovah! der biſt Du!

    \voice[Das Volk]
    O Finſterniß! o Nacht! iſt keine Rettung da?
    Verflucht ſey Kaiphas! verflucht ſey Golgatha!

    \voice[Der Hauptmann]
    Cäcil, auf eile nach Jeruſalem,
    die Schaaren zu verſtärken! —

    \voice[Ein Soldat]
    \hspace*{3cm}Herr! wer findet eine Bahn
    in dieſer dicken Nacht?

    \voice[Der Hauptmann]
    \hspace*{3cm}Angſt, Zittern, Furcht und Schrecken!
    Wie? ſollen die mich auch, wie dieſes Volk, bedecken?

    \voice[Ein andrer aus dem Volke]
    Ach! es iſt aus mit uns? Was haben wir gethan?

    \voice[Die römiſchen Soldaten]
    Helft, Götter! helft! Verdammte Zaubernacht!
    Hier wird man ohne Feind im Finſtern umgebracht.

    \voice[Maria]
    Ach ſtütze mich, Johann! mein neuer Sohn!
    Mich! die nicht mehr vor Schrecken ſtehen kann.
    Welch Unglück prophezeyht die traurge Dunkelheit?
    Die bange Finſterniß, die mich verhindert,
    mein Heil und meinen Sohn im Sterben anzuſehn?
    Sein ſanfter Blick, der ſonſt in tiefſter Traurigkeit
    aufs ſüſſeſte die bängſten Schmerzen mindert,
    fließt nicht mehr in mein Herz. Sein Ohr hört nicht mein Flehn.
    Ach Sohn! ach Gott! ach! lebſt du, oder nicht?
    Vielleicht, daß nun dein Aug’ im Tod’ entſchlummernd bricht!
  \end{movement}

  % \begin{movement}{}
  %   \voice[]
  % \end{movement}

  % \begin{movement}{}
  %   \voice[]
  % \end{movement}

  % \begin{movement}{}
  %   \voice[]
  % \end{movement}

  % \begin{movement}{}
  %   \voice[]
  % \end{movement}

  % \begin{movement}{}
  %   \voice[]
  % \end{movement}

  % \begin{movement}{}
  %   \voice[]
  % \end{movement}
}

\eesScore

\end{document}
