\documentclass{ees}

\begin{document}

\eesTitlePage

\eesCriticalReport{
  – & –   & A       & S 2 may be replaced by A throughout the piece \\
  3 & 54  & vl 2    & 7th \eighthNote\ in \A1: e′8 \\
  5 & 52  & vla     & 2nd \eighthNote\ in \A1: \flat b8 \\
    & 242 & Eidli   & 2nd/3rd \eighthNote\ in \A1: c″8–d″16–\flat e″16 \\
}

Eidli (S = Canto I)
Maria (A = Canto II)
Eine aus den Weibern (S/A)
Ein andrer aus dem Volke (T)
Jeſus (T)
Johannes (T)
Lazarus (T)
Einer aus dem Volke (B)
Der römiſche Hauptmann (B)


\eesToc{
  \begin{movement}{wosindwir}
    \voice[ Das Volk]
    Wo ſind wir? wir? die Furcht und Schrecken
    und grauſe Dunkelheit bedecken?
    Wir fühlen des Todes erſchreckliche Nacht!
    O! deine Rache, Jehovah! erwacht!
    Du zürneſt — du rächeſt — wen? dieſen Verbrecher? —
    Nein, Herr! du biſt nicht des Gekreuzigten Rächer.
    Du ſtrafeſt? — wer ſündigte? — tödtende Nacht!
    Frolocke, Mizraim! dein Schickſal erwacht!
    Der Herr hat ſein Juda dir ähnlich gemacht!
  \end{movement}

  \begin{movement}{werduauch}
    \voice[Einer aus dem Volke]
    Wer du auch ſeyſt, den man dort angeheftet hat,
    wer du auch ſeyſt, Mesſias, oder ein Verräther,
    Gott, oder auch ein Miſſethäter!
    Du biſt der Mann, der nun, o ſchwarze That!
    die grauſe Finſterniß erſchaffen.
    Du tödteſt uns mit neuerfundnen Waffen!
    O! richteten die Prieſter auch wohl recht? —

    \voice[Ein andrer aus dem Volke]
    Unglücklichs Land! unſeliges Geſchlecht!
    Verloren ſind wir ſchon. Es nähert ſich die Stunde
    des Unglücks, des Verderbens,
    des Todes, des niemals erhörten Sterbens.
    Verdammter Rath! durch dich geht Gottes Volk zu Grunde.
    Ohnfehlbar iſt er der Mesſias, und unſchuldig.
    Hoch rächet er ſich nun. —

    \voice[Die Weiber und die Gläubigen]
    \hspace*{3cm}Ja! Heil! Mesſias! Gott!
    Unſchuldig biſt du, Herr! Du, deiner Feinde Spott!

    \voice[Eine aus den Weibern]
    Ja! dies bezeugt die nun erblaßte Sonne.
    Du ſtirbſt. Es ſtirbt mit dir auch unſre Wonne
    und unſre Hoffnung —

    \voice[Die Weiber und die Gläubigen]
    \hspace*{3cm}Herr! nun rächet dich der Himmel!

    \voice[Einer aus dem Volke]
    Verdammtes Urtheil! Pilatus! Prieſter! Rath!

    \voice[Das Volk]
    Man tödte ſie, die Schuldgen dieſer That!

    \voice[Der römiſche Hauptmann]
    Welch raſendes Getümmel!
    Volk, voller Angſt! welch eine finſtre Wut!
    Wie? foderſt du noch deiner Vorgeſetzten Blut?
    Soldaten! wehrt dem Aufruhr! —

    \voice[Einer aus dem Volke]
    \hspace*{3cm}Halt!
    Nicht du, nicht deine Schaar hemmt Eifer und Gewalt;
    dies thut ein ſtärkrer Arm, ein höhrer Ruf.
    Die Finſterniß, durch die kein Strahl des Lichtes dringet,
    hält uns zurück. Die iſts, die unſer Wüten zwinget.
    Doch nein! vielmehr der, der ſie ſchuf.
    Und, o Gewaltiger! Jehovah! der biſt Du!

    \voice[Das Volk]
    O Finſterniß! o Nacht! iſt keine Rettung da?
    Verflucht ſey Kaiphas! verflucht ſey Golgatha!

    \voice[Der Hauptmann]
    Cäcil, auf eile nach Jeruſalem,
    die Schaaren zu verſtärken! —

    \voice[Ein Soldat]
    \hspace*{3cm}Herr! wer findet eine Bahn
    in dieſer dicken Nacht?

    \voice[Der Hauptmann]
    \hspace*{3cm}Angſt, Zittern, Furcht und Schrecken!
    Wie? ſollen die mich auch, wie dieſes Volk, bedecken?

    \voice[Ein andrer aus dem Volke]
    Ach! es iſt aus mit uns? Was haben wir gethan?

    \voice[Die römiſchen Soldaten]
    Helft, Götter! helft! Verdammte Zaubernacht!
    Hier wird man ohne Feind im Finſtern umgebracht.

    \voice[Maria]
    Ach ſtütze mich, Johann! mein neuer Sohn!
    Mich! die nicht mehr vor Schrecken ſtehen kann.
    Welch Unglück prophezeyht die traurge Dunkelheit?
    Die bange Finſterniß, die mich verhindert,
    mein Heil und meinen Sohn im Sterben anzuſehn?
    Sein ſanfter Blick, der ſonſt in tiefſter Traurigkeit
    aufs ſüſſeſte die bängſten Schmerzen mindert,
    fließt nicht mehr in mein Herz. Sein Ohr hört nicht mein Flehn.
    Ach Sohn! ach Gott! ach! lebſt du, oder nicht?
    Vielleicht, daß nun dein Aug’ im Tod’ entſchlummernd bricht!
  \end{movement}

  \begin{movement}{otoedtet}
    \voice[Maria]
    O! tödtet mich nur auch, vermeßne Schaaren!
    Mein Sohn iſt todt: ſo tödtet nun auch mich!

    \voice[Jeſus]
    Maria!

    \voice[Maria]
    Herr! ſo rufſt du warnend mir noch zu?
    Du lebſt, mein Heil! du ſegneſt meine Ruh!
    Doch ach! was ſoll mir dieſes Leben nützen?
    Wer wird mich nun im höhern Alter ſtützen?
    Was iſt Maria ohne dich?
    Ach Gott! was muß die Mutter deines Sohns erfahren?
    Mein Schmerz erwacht, der Troſt verlieret ſich.
    Drum tödtet mich nur auch, vermeßne Schaaren!
    Mein Sohn iſt todt: ſo tödtet nun auch mich!
  \end{movement}

  \begin{movement}{achheiland}
    \voice[Maria]
    Ach! Heiland! Sohn! ach tröſte mich! —
    Mich tröſte mit der Stimme deines Mundes! —
    Dein Wort iſt ſüſſer noch, als Milch und Honigſeim. —
    Und ach! was bin ich ohne dich?
    Und biſt du denn — doch ja! du biſt
    der Stifter eines neuen Bundes:
    O Herr! ſo ſieh auf uns, auf unſre bangen Schmerzen!
    Ergieße deinen Troſt, der unſer Labſal iſt,
    in unſre tiefgebeugte Herzen!
    Sieh durch die dicke Nacht ins Dunkle meiner Seelen,
    wo tauſend Martern mich, mich deine Mutter, quälen.
    Erlöſer! — Seelenfreund! — Ach Sohn! — Die Worte fehlen mir —
    die Zunge ſtarrt — ach! —

    \voice[Johannes]
    \hspace*{3cm}Gott! mein Herr! mein Meiſter!
    Im Geiſte nah ich mich zu dir.
    Anbetenswürdiger! Du kenneſt meine Triebe:
    Ich bin durchaus entflammt von deiner heilgen Liebe.
    Herr! dieſe dunklen Stunden,
    die deine Mörder ſelbſt durchdringend rühren,
    umnebeln auch mein Herz, und tödten mich mit Wunden,
    die ganz unheilbar ſind. Herr! dich, dich zu verlieren —
    Gedanke! ſchwärzer, als die Nacht, die uns umhüllt! —
    Herr! der Verluſt wirft mich in eine Gruft von Schrecken,
    aus der mich, auſſer du, nichts wird, nichts kann erwecken. —
    Wiewohl! — Vielleicht erwachſt du wieder! —
    die Hoffnung! — würde ſie erfüllt! —
    O! Freude herrſcht ſie ſchon durch alle meine Glieder. —
    Ja! Herr, du wirſt, nach dreyer Tage Friſt,
    du ſagſt es ſelbſt, aufs neu erwachen,
    und dieſen Tempel, der nun faſt zerſtöret iſt,
    aufs neue baun, und zwar weit herrlicher.
    Du wirſt ihn auch weit unvergänglicher,
    weit ſchöner machen,
    als er zuvor geweſen. — Heiland! — Ja, es iſt gewiß!
    Dann weichen Furcht und Angſt; dann weicht die Finſterniß.

    \voice[Eidli]
    Mein Lazarus! mein mir von Gott geſchaffner Freund!
    Mein Liebſter! ſäheſt du, wie dieſes Auge weint!
    Es weint — ach Gott! — Still, wie der Tod,
    werd ich durch die Empfindung dieſer Noth,
    die alles Land bedeckt.
    Nicht, weil die Finſterniß die bange Welt erſchreckt,
    nein, Freund! weil unſer Helfer ſtirbt;
    da der Erwecker ſtirbt, der dir und mir das Leben
    aufs neu ehmals allmächtig hat gegeben,
    als unſer Körper kalt, erſtarrt, entſeelet war.

    \voice[Lazarus]
    Mich macht mein Kummer ſtumm, — die matten Glieder beben, —
    das Mark in dem Gebein gerinnt, das Blut ſteht ſtill —
    wird ſtarr und kalt — wie ehmals mir geſchah,
    als mich die Lebenskraft verließ,
    und dieſem Leib der Tod den Geiſt entriß. —
    Mit Ach und Weh erfülltes Golgatha! —
    Ich weiß nicht, was ich denk und will —
    mein weinend Auge bricht —
    und würde ſchon die Finſterniß verſchwinden:
    So würde mein gebrochnes Augenlicht
    doch keinen Strahl des Lichts empfinden.
    Mein Herr! mein Gott! Mesſias! Tröſter! Ach! —
    Nie, Juda! drückte dich ein ähnlich ſtarker Schmerz —
    durchſchaudernd überſtrömt das ängſtlichbängſte Herz
    ein kalter Thränenbach —
    du ſtirbſt —

    \voice[Eidli]
    \hspace*{3cm}Du ſtirbſt! wer kann daran gedenken?

    \voice[Lazarus]
    Du ſtirbſt — wen muß dieß nicht erſchütterndtödlich kränken?

    \voice[Eidli und Lazarus]
    Wie? mußten wir denn darum auferſtehen,
    um hier auf Golgatha, Herr! deinen Tod zu ſehen?
  \end{movement}

  \begin{movement}{duherrscher}
    \voice[Eidli]
    Du Herrſcher über Tod und Leben!
    Nimm hin das, was du mir gegeben.
    Stirb ohne deine Eidli nicht!

    \voice[Lazarus]
    Zu meiner Gruft, der Freunde Schrecken,
    kamſt du, mein Heil, mich aufzuwecken,
    zu ſehn, wie itzt dein Auge bricht.

    \voice[Eidli und Lazarus]
    O Jeſu! Freund! dieß Leben ſchenkſt du mir:
    Nimm dein Geſchenk im Tode doch zu dir!

    \voice[Eidli]
    Zwar, Freund! bindt uns das Band der reinſten Liebe;
    doch wie? vergnügen irrdſche Triebe,
    wenn mir mein Heil, mein Jeſus, ſtirbt?

    \voice[Lazarus]
    Zwar, Freundinn! fand in deinem frommen Blicke
    ehmals mein Aug’ ein edles Glücke;
    doch was iſt das, wenn Jeſus ſtirbt?

    \voice[Eidli und Lazarus]
    Ja, Herr! nimm Neigung, Leben, alles hin!
    Wohl mir, wenn ich im Tode bey dir bin!
  \end{movement}

  \begin{movement}{wieliebens}
    \voice[Johannes]
    Wie liebenswürdig ſind die Triebe
    der göttlichen, der heilgen Jeſusliebe,
    die euch und mich zu dieſem Kreuze zieht!
    In dieſer Finſterniß, da Alles ſeufzt und weinet,
    da uns kein Troſt, kein Hoffnungslicht erſcheinet,
    da auch dem Auge ſelbſt der Hoffnungsquell entflieht:
    Wer wollt’ um den Verluſt nicht kläglich wimmernd zagen?
    Wer wollte nicht, mein Heil! um dich mit Jammer klagen?
    Mit Jammer! der ſogar die Wut der Feind’ erſtickt,
    indem die Todesnacht ihr Raſen unterdrückt.
    O grauſame, o fürchterliche Schatten!
    Entzieht den Augen nicht das, was den Geiſt erquickt!
    Entzieht den Augen nicht den Urquell meiner Liebe,
    den heilgen Gegenſtand der undurchforſchten Triebe!
    Stärkt nur ein einzigmal mich Matten
    mit einem dunklen Licht. Denn ach! das größte Leiden,
    das Schrecklichſte iſt dies: den heilgen Blick zu meiden,
    von dem mein Leben ſtammt.
    O! flöſſe dieſer nur in das gepreßte Herz:
    O! ſo ertrug es willig ſeinen Schmerz!
    So aber iſt es, Gott! zum ſchwärzſten Gram verdammt.
  \end{movement}

  % \begin{movement}{}
  %   \voice[]
  % \end{movement}

  % \begin{movement}{}
  %   \voice[]
  % \end{movement}
}

\eesScore

\end{document}
