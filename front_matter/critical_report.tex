\documentclass{ees}

\begin{document}

\eesTitlePage

\eesCriticalReport{
  –  & –   & A       & S 2 may be replaced by A throughout the piece \\
  3  & 54  & vl 2    & 7th \eighthNote\ in \A1: e′8 \\
  5  & 52  & vla     & 2nd \eighthNote\ in \A1: \flat b8 \\
     & 242 & Eidli   & 2nd/3rd \eighthNote\ in \A1: c″8–d″16–\flat e″16 \\
  7  & 71  & Johannes & 1st \halfNote\ in \A1: \flat g′4–\flat g′8–\flat g′8 \\
  9  & 70  & T       & 1st \halfNote\ in \A1: f2 \\
     & 117 & B       & 8th \sixteenthNote\ in \A1: \flat f16 \\
  11 & 1   & soli    & 5th \eighthNote\ missing in \A1 \\
}

Eidli (S = Canto I)
Maria (A = Canto II)
Eine aus den Weibern (S/A)
Ein andrer aus dem Volke (T)
Jeſus (T)
Johannes (T)
Lazarus (T)
Einer aus dem Volke (B)
Der römiſche Hauptmann (B)


\eesToc{
  \begin{movement}{wosindwir}
    \voice[Das Volk]
    Wo ſind wir? wir? die Furcht und Schrecken
    und grauſe Dunkelheit bedecken?
    Wir fühlen des Todes erſchreckliche Nacht!
    O! deine Rache, Jehovah! erwacht!
    Du zürneſt — du rächeſt — wen? dieſen Verbrecher? —
    Nein, Herr! du biſt nicht des Gekreuzigten Rächer.
    Du ſtrafeſt? — wer ſündigte? — tödtende Nacht!
    Frolocke, Mizraim! dein Schickſal erwacht!
    Der Herr hat ſein Juda dir ähnlich gemacht!
  \end{movement}

  \begin{movement}{werduauch}
    \voice[Einer aus dem Volke]
    Wer du auch ſeyſt, den man dort angeheftet hat,
    wer du auch ſeyſt, Mesſias, oder ein Verräther,
    Gott, oder auch ein Miſſethäter!
    Du biſt der Mann, der nun, o ſchwarze That!
    die grauſe Finſterniß erſchaffen.
    Du tödteſt uns mit neuerfundnen Waffen!
    O! richteten die Prieſter auch wohl recht? —

    \voice[Ein andrer aus dem Volke]
    Unglücklichs Land! unſeliges Geſchlecht!
    Verloren ſind wir ſchon. Es nähert ſich die Stunde
    des Unglücks, des Verderbens,
    des Todes, des niemals erhörten Sterbens.
    Verdammter Rath! durch dich geht Gottes Volk zu Grunde.
    Ohnfehlbar iſt er der Mesſias, und unſchuldig.
    Hoch rächet er ſich nun. —

    \voice[Die Weiber und die Gläubigen]
    \hspace*{3cm}Ja! Heil! Mesſias! Gott!
    Unſchuldig biſt du, Herr! Du, deiner Feinde Spott!

    \voice[Eine aus den Weibern]
    Ja! dies bezeugt die nun erblaßte Sonne.
    Du ſtirbſt. Es ſtirbt mit dir auch unſre Wonne
    und unſre Hoffnung —

    \voice[Die Weiber und die Gläubigen]
    \hspace*{3cm}Herr! nun rächet dich der Himmel!

    \voice[Einer aus dem Volke]
    Verdammtes Urtheil! Pilatus! Prieſter! Rath!

    \voice[Das Volk]
    Man tödte ſie, die Schuldgen dieſer That!

    \voice[Der römiſche Hauptmann]
    Welch raſendes Getümmel!
    Volk, voller Angſt! welch eine finſtre Wut!
    Wie? foderſt du noch deiner Vorgeſetzten Blut?
    Soldaten! wehrt dem Aufruhr! —

    \voice[Einer aus dem Volke]
    \hspace*{3cm}Halt!
    Nicht du, nicht deine Schaar hemmt Eifer und Gewalt;
    dies thut ein ſtärkrer Arm, ein höhrer Ruf.
    Die Finſterniß, durch die kein Strahl des Lichtes dringet,
    hält uns zurück. Die iſts, die unſer Wüten zwinget.
    Doch nein! vielmehr der, der ſie ſchuf.
    Und, o Gewaltiger! Jehovah! der biſt Du!

    \voice[Das Volk]
    O Finſterniß! o Nacht! iſt keine Rettung da?
    Verflucht ſey Kaiphas! verflucht ſey Golgatha!

    \voice[Der Hauptmann]
    Cäcil, auf eile nach Jeruſalem,
    die Schaaren zu verſtärken! —

    \voice[Ein Soldat]
    \hspace*{3cm}Herr! wer findet eine Bahn
    in dieſer dicken Nacht?

    \voice[Der Hauptmann]
    \hspace*{3cm}Angſt, Zittern, Furcht und Schrecken!
    Wie? ſollen die mich auch, wie dieſes Volk, bedecken?

    \voice[Ein andrer aus dem Volke]
    Ach! es iſt aus mit uns? Was haben wir gethan?

    \voice[Die römiſchen Soldaten]
    Helft, Götter! helft! Verdammte Zaubernacht!
    Hier wird man ohne Feind im Finſtern umgebracht.

    \voice[Maria]
    Ach ſtütze mich, Johann! mein neuer Sohn!
    Mich! die nicht mehr vor Schrecken ſtehen kann.
    Welch Unglück prophezeyht die traurge Dunkelheit?
    Die bange Finſterniß, die mich verhindert,
    mein Heil und meinen Sohn im Sterben anzuſehn?
    Sein ſanfter Blick, der ſonſt in tiefſter Traurigkeit
    aufs ſüſſeſte die bängſten Schmerzen mindert,
    fließt nicht mehr in mein Herz. Sein Ohr hört nicht mein Flehn.
    Ach Sohn! ach Gott! ach! lebſt du, oder nicht?
    Vielleicht, daß nun dein Aug’ im Tod’ entſchlummernd bricht!
  \end{movement}

  \begin{movement}{otoedtet}
    \voice[Maria]
    O! tödtet mich nur auch, vermeßne Schaaren!
    Mein Sohn iſt todt: ſo tödtet nun auch mich!

    \voice[Jeſus]
    Maria!

    \voice[Maria]
    Herr! ſo rufſt du warnend mir noch zu?
    Du lebſt, mein Heil! du ſegneſt meine Ruh!
    Doch ach! was ſoll mir dieſes Leben nützen?
    Wer wird mich nun im höhern Alter ſtützen?
    Was iſt Maria ohne dich?
    Ach Gott! was muß die Mutter deines Sohns erfahren?
    Mein Schmerz erwacht, der Troſt verlieret ſich.
    Drum tödtet mich nur auch, vermeßne Schaaren!
    Mein Sohn iſt todt: ſo tödtet nun auch mich!
  \end{movement}

  \begin{movement}{achheiland}
    \voice[Maria]
    Ach! Heiland! Sohn! ach tröſte mich! —
    Mich tröſte mit der Stimme deines Mundes! —
    Dein Wort iſt ſüſſer noch, als Milch und Honigſeim. —
    Und ach! was bin ich ohne dich?
    Und biſt du denn — doch ja! du biſt
    der Stifter eines neuen Bundes:
    O Herr! ſo ſieh auf uns, auf unſre bangen Schmerzen!
    Ergieße deinen Troſt, der unſer Labſal iſt,
    in unſre tiefgebeugte Herzen!
    Sieh durch die dicke Nacht ins Dunkle meiner Seelen,
    wo tauſend Martern mich, mich deine Mutter, quälen.
    Erlöſer! — Seelenfreund! — Ach Sohn! — Die Worte fehlen mir —
    die Zunge ſtarrt — ach! —

    \voice[Johannes]
    \hspace*{3cm}Gott! mein Herr! mein Meiſter!
    Im Geiſte nah ich mich zu dir.
    Anbetenswürdiger! Du kenneſt meine Triebe:
    Ich bin durchaus entflammt von deiner heilgen Liebe.
    Herr! dieſe dunklen Stunden,
    die deine Mörder ſelbſt durchdringend rühren,
    umnebeln auch mein Herz, und tödten mich mit Wunden,
    die ganz unheilbar ſind. Herr! dich, dich zu verlieren —
    Gedanke! ſchwärzer, als die Nacht, die uns umhüllt! —
    Herr! der Verluſt wirft mich in eine Gruft von Schrecken,
    aus der mich, auſſer du, nichts wird, nichts kann erwecken. —
    Wiewohl! — Vielleicht erwachſt du wieder! —
    die Hoffnung! — würde ſie erfüllt! —
    O! Freude herrſcht ſie ſchon durch alle meine Glieder. —
    Ja! Herr, du wirſt, nach dreyer Tage Friſt,
    du ſagſt es ſelbſt, aufs neu erwachen,
    und dieſen Tempel, der nun faſt zerſtöret iſt,
    aufs neue baun, und zwar weit herrlicher.
    Du wirſt ihn auch weit unvergänglicher,
    weit ſchöner machen,
    als er zuvor geweſen. — Heiland! — Ja, es iſt gewiß!
    Dann weichen Furcht und Angſt; dann weicht die Finſterniß.

    \voice[Eidli]
    Mein Lazarus! mein mir von Gott geſchaffner Freund!
    Mein Liebſter! ſäheſt du, wie dieſes Auge weint!
    Es weint — ach Gott! — Still, wie der Tod,
    werd ich durch die Empfindung dieſer Noth,
    die alles Land bedeckt.
    Nicht, weil die Finſterniß die bange Welt erſchreckt,
    nein, Freund! weil unſer Helfer ſtirbt;
    da der Erwecker ſtirbt, der dir und mir das Leben
    aufs neu ehmals allmächtig hat gegeben,
    als unſer Körper kalt, erſtarrt, entſeelet war.

    \voice[Lazarus]
    Mich macht mein Kummer ſtumm, — die matten Glieder beben, —
    das Mark in dem Gebein gerinnt, das Blut ſteht ſtill —
    wird ſtarr und kalt — wie ehmals mir geſchah,
    als mich die Lebenskraft verließ,
    und dieſem Leib der Tod den Geiſt entriß. —
    Mit Ach und Weh erfülltes Golgatha! —
    Ich weiß nicht, was ich denk und will —
    mein weinend Auge bricht —
    und würde ſchon die Finſterniß verſchwinden:
    So würde mein gebrochnes Augenlicht
    doch keinen Strahl des Lichts empfinden.
    Mein Herr! mein Gott! Mesſias! Tröſter! Ach! —
    Nie, Juda! drückte dich ein ähnlich ſtarker Schmerz —
    durchſchaudernd überſtrömt das ängſtlichbängſte Herz
    ein kalter Thränenbach —
    du ſtirbſt —

    \voice[Eidli]
    \hspace*{3cm}Du ſtirbſt! wer kann daran gedenken?

    \voice[Lazarus]
    Du ſtirbſt — wen muß dieß nicht erſchütterndtödlich kränken?

    \voice[Eidli und Lazarus]
    Wie? mußten wir denn darum auferſtehen,
    um hier auf Golgatha, Herr! deinen Tod zu ſehen?
  \end{movement}

  \begin{movement}{duherrscher}
    \voice[Eidli]
    Du Herrſcher über Tod und Leben!
    Nimm hin das, was du mir gegeben.
    Stirb ohne deine Eidli nicht!

    \voice[Lazarus]
    Zu meiner Gruft, der Freunde Schrecken,
    kamſt du, mein Heil, mich aufzuwecken,
    zu ſehn, wie itzt dein Auge bricht.

    \voice[Eidli und Lazarus]
    O Jeſu! Freund! dieß Leben ſchenkſt du mir:
    Nimm dein Geſchenk im Tode doch zu dir!

    \voice[Eidli]
    Zwar, Freund! bindt uns das Band der reinſten Liebe;
    doch wie? vergnügen irrdſche Triebe,
    wenn mir mein Heil, mein Jeſus, ſtirbt?

    \voice[Lazarus]
    Zwar, Freundinn! fand in deinem frommen Blicke
    ehmals mein Aug’ ein edles Glücke;
    doch was iſt das, wenn Jeſus ſtirbt?

    \voice[Eidli und Lazarus]
    Ja, Herr! nimm Neigung, Leben, alles hin!
    Wohl mir, wenn ich im Tode bey dir bin!
  \end{movement}

  \begin{movement}{wieliebens}
    \voice[Johannes]
    Wie liebenswürdig ſind die Triebe
    der göttlichen, der heilgen Jeſusliebe,
    die euch und mich zu dieſem Kreuze zieht!
    In dieſer Finſterniß, da Alles ſeufzt und weinet,
    da uns kein Troſt, kein Hoffnungslicht erſcheinet,
    da auch dem Auge ſelbſt der Hoffnungsquell entflieht:
    Wer wollt’ um den Verluſt nicht kläglich wimmernd zagen?
    Wer wollte nicht, mein Heil! um dich mit Jammer klagen?
    Mit Jammer! der ſogar die Wut der Feind’ erſtickt,
    indem die Todesnacht ihr Raſen unterdrückt.
    O grauſame, o fürchterliche Schatten!
    Entzieht den Augen nicht das, was den Geiſt erquickt!
    Entzieht den Augen nicht den Urquell meiner Liebe,
    den heilgen Gegenſtand der undurchforſchten Triebe!
    Stärkt nur ein einzigmal mich Matten
    mit einem dunklen Licht. Denn ach! das größte Leiden,
    das Schrecklichſte iſt dies: den heilgen Blick zu meiden,
    von dem mein Leben ſtammt.
    O! flöſſe dieſer nur in das gepreßte Herz:
    O! ſo ertrug es willig ſeinen Schmerz!
    So aber iſt es, Gott! zum ſchwärzſten Gram verdammt.
  \end{movement}

  \begin{movement}{fliesstzitternde}
    \voice[Johannes]
    Fließt, zitternde Thränen, entdecket mein Leiden!
    Dich ſoll ich, o zärtlichſter Gegenſtand, meiden.
    Es hat mich dieß tödtende Dunkel der Nacht
    auch um deine Blicke, mein Liebſter! gebracht.

    Und obſchon die Hoffnung, zwar dunkler, als Nächte,
    mich Weinenden tröſtend zu ſtärken gedächte;
    und obſchon ein Fünkchen des Lichtes erwacht:
    So tödtet doch Hoffnung und Alles die Nacht.
  \end{movement}

  \begin{movement}{dochherz}
    \voice[Johannes]
    Doch Herz! von manchen Wiederſprüchen voll!
    Verzweifle nicht! bald, bald vergeht die Nacht;
    bald, bald erſcheint die Zeit, da Jeſus dir erwacht!
    Wie? hab ich nicht zuvor, ich ſelbſt? daran gedacht?
    Ja, dieſer Troſt hat ſchon mein Herz erhellt gemacht.
    Was pochſt du Herz? Sein Wort beſtimmt gewiß
    das, was geſchehen ſoll,
    und bald entweicheſt du, grauſame Finſterniß!
    Doch, bis dieß Wort, Herr! zur Erfüllung wird gebracht,
    ſo lange will ich, Freunde, mit euch weinen;
    denn keine Thränen ſind zu viel.
    Getroſt! der Schmerz währt eine kurze Zeit:
    ſein liebreich Angeſicht wird glänzend bald erſcheinen,
    zertheilt die Angſt, ſo wie die Dunkelheit,
    und ſetzt dem kurzen Gram ein froherreichtes Ziel.
  \end{movement}

  \begin{movement}{omatter}
    \voice[Die Weiber und die Gläubigen]
    O matter Troſt bey ſo viel Kummer!
    Er liegt ja ſchon im Todesſchlummer,
    nicht einen Seufzer höret man.
    Sein Aug iſt dunkler, als die Lüfte.
    O Tod! eröffne deine Grüfte!
    Verſchling ein Volk, das nicht mehr leben kann.

    \voice[Maria]
    Mein Sohn! hör deines Volks betrübtes Seufzen an!
    Verlaſſen ſtehn die Deinen.
    Mein Sohn! erbarmt dich nicht der Mutter ängſtlich Weinen?
    Biſt du nicht mehr zu unſrer Rettung nah?
    Ein jeder ſteht und harrt, auf was? nur aufs Verderben.
    Ein jeder ſeufzt und hofft, allhier auf Golgatha
    zum wenigſten mit dir zu ſterben.
    Kein einzger wünſchet, dich zu überleben.

    \voice[Eidli]
    Mein Wunſch, mein ganzer Wunſch ſtimmt damit überein.

    \voice[Die Weiber und die Gläubigen]
    Kann wohl ein andrer Wunſch uns Armen möglich ſeyn?

    \voice[Eidli]
    Das, was er mir ſonſt gab, mein wundervolles Leben,
    verlang ich nicht mehr zu genieſſen.
    Es konnte blos aus ihm die Luft des Lebens flieſſen.
    Was Leben! o Verluſt! der Retter ſelbſt iſt todt —
    ach todt — ohnfehlbar iſt ſein Augenlicht erblaßt —

    \voice[Lazarus]
    Ohnfehlbar iſt mein Heil, mein Auferwecker todt.
    Nun fühl ich erſt des Lebens ſchwere Laſt. —
    Doch Herr! — und lebſt du noch? — Du? — Quell von meinem Leben —
    Wie? — oder haſt du ſchon den Geiſt — ach! aufgegeben? —

    \voice[Die Weiber und die Gläubigen]
    Ach! er iſt todt! — Ja! — dieſe Schreckensnacht
    hat uns um unſern Herrn, um alle Ruh gebracht!

    \voice[Johannes]
    Getroſt! verzaget nicht! Erwartet größre Thaten!
    Schon hab ichs euch geſagt. Ich bin noch mehr bewegt,
    indem ein höhrer Trieb des Troſtes mich erregt. —
    Weiſſagend ruf ich euch in ſelger Hoffnung zu:
    Es eilet im Triumph der Held nunmehr zur Ruh.
    Ihm ſoll zum Wohl der Welt der größte Sieg gerathen!

    \voice[Maria]
    Ich weiß nicht, welcher Trieb dem Ausſpruch Beyfall giebt.
    Mein theurſter Sohn, mein Gott, der mich ſo ſehr geliebt,
    wird Marter, Kreuz und Tod und Grab
    zu Quellen eines Heils, das uns verewget, machen.
    Ihr Gläubgen! ſeyd getroſt! legt Furcht und Trauren ab!

    \voice[Die Weiber und die Gläubigen]
    Ach! wäre dieß gewiß! ach! mögt er bald erwachen!
  \end{movement}

  \begin{movement}{selge}
    \voice[Maria]
    Selge Hoffnung! wie ſtärkſt du die Glieder!
    Glieder! ſtarr von Furcht und Tod.
    Süſſer Jeſus! ja, wir leben wieder,
    und erwarten ſtill das Ende dieſer Noth.
    Held! ja, dir wird bald der Sieg gerathen.
    Uns beleben ſchon die Früchte deiner Thaten!

    \voice[Die Weiber und die Gläubigen]
    Held! ja, dir wird bald der Sieg gerathen.
    Uns beleben ſchon die Früchte deiner Thaten!
  \end{movement}

  \begin{movement}{wasfuer}
    \voice[Der römiſche Hauptmann]
    Was für ein Volk iſt das? Itzt, da die Finſterniß
    die Welt ſo furchtbar ſchrecket,
    und eine ſchwarze Nacht
    uns bis aufs Äuſſerſte der bängſten Furcht gebracht, —
    (O! eine Finſterniß! die ſelbſt den Tag verdecket!)
    faſt zittr’ ich! — Itzt da dieſe Schreckensnacht
    auf nieerhörte Art den Himmel ganz verdunkelt,
    daß auch kein einzger Stern, kein mattes Wölkchen funkelt,
    daß man nicht Himmel, Luft, nicht Erd und Menſchen ſieht,
    nicht ſeinen Nachbar merkt, nicht Schwerdt und Harniſch blinken,
    da, wenn man fällt, man in den Abgrund denkt zu ſinken,
    da man vor Angſt ſich ſelbſt und alles flieht,
    und doch ſich fürchtet, zu entfliehen,
    da auch die Kühnſten ſich dem Schrecken nicht entziehen,
    die Helden ſtill und heimlich ächzen,
    von tiefen Seufzern alle Gaumen lechzen:
    Wie? da erholen ſich der ſchwachen Weiber Herzen,
    und ſiegen über uns und über Angſt und Schmerzen? —
    Zertheile Jupiter! — doch nein! —
    hat Jupiter wohl Theil an ſo viel Wundern? —
    Wie? — Nicht ein Jupiter! — Es muß ein höhres Weſen ſeyn. —
    Ein Sterbender! — um den die Sonne ſelbſt verſchwindet —
    Und Weiber — die im Schrecken herzhaft werden —
    furchtſame Männer — ſonſt die feigſten dieſer Erden,
    von einem ſchwachen Strahl der Hoffnung ſtark entzündet —
    dort ein Gekreuzigter — von Geiſſeln ganz zerriſſen,
    voll Wunden ohne Zahl, matt, kraftlos, todt und kalt, —
    dem väterlichen Volk zum Scheuſal aufgeſtellt —
    und kurz: ein Schimpf, ein Spott der Welt —
    der, der flößt noch mit unbegreiflicher Gewalt
    dem Häuflein, das ihn ehrt, Geduld und Hoffnung ein? —
    Der ſoll ein Troſt, ein Schutz, und ein Erretter ſeyn? —
  \end{movement}

  \begin{movement}{wobinich}
    \voice[Der römiſche Hauptmann]
    Wo bin ich? — Was rühret mich? —
    Verwirrung! — Entſetzen! — Nacht und Tod!
    Und Sonne! — wo biſt du? — verſchwunden —
    und was durchſchaudert mich? —
    Sinds Schmerzen? — ſinds Wunden? —
    Nein! — Herz! was bebſt du? — welche Noth! —
    In Finſterniß — erbärmliche Geſtalt! —
    Natur! — woher ſo ſchreckliche Gewalt? —
    Gekreuzigter! — vielleicht durch dich?
  \end{movement}

  \begin{movement}{dochtraeum}
    \voice[Der römiſche Hauptmann]
    Doch träum ich? — Wird es Tag? — Hilf Himmel! welcher Schein!
    Welch Wunder! was geſchieht? — Die Sonne glänzet wieder?
    die Finſterniß entweicht, ſo plötzlich, als ſie kam!

    \voice[Das Volk]
    O Wunder!

    \voice[Eine aus den Weibern]
    \hspace*{3cm} Jeſus lebt! lebt auch erſtorbne Glieder!

    \voice[Die Weiber und die Gläubigen]
    Er lebt!

    \voice[Einer aus dem Volke]
    \hspace*{3cm}Es ſterben Furcht und Gram!
    Getroſt! Gott iſt verſöhnt, ſein Zorn geſtillt.
    Das Urtheil iſt nunmehr beſtätigt, bald erfüllt.
    Gott foderte des Nazareners Blut.

    \voice[Das Volk]
    Wir Feigen zitterten aus bloßem Mißverſtande.
    Die Prieſter irrten nicht.

    \voice[Johannes]
    \hspace*{3cm}O blinde Wut!
    Unſelges Volk! der Menſchheit Schande!
    Verſtockter Schwarm! den man, kaum da die Nacht entflieht,
    in vorger Raſerey ſich tollkühn wälzen ſieht!
    Den des Unſchuldigen halbtodtes Leben kränket,
    das er mit neuem Grimm nun zu verfolgen denket.
    Zwar Jeſus lebt. Doch kann das wohl ein Leben heiſſen,
    das Wunden, Schmerz und Schmach und Todesangſt zerreiſſen?
    Seht, Gottesmörder! Seht nur ſeine Martern an! —
    ſeht ſie! — erbebt! — ſeufzt, ächzt: was haben wir gethan?

    \voice[Jeſus]
    Mein Gott! mein Gott! warum haſt du mich verlaſſen?

    \voice[Die Weiber und die Gläubigen]
    Ach! banger Ruf! wer kann die Deutung faſſen?

    \voice[Johannes]
    Jehovah! Gott! wie ſeufzt das Lamm!
    das Lamm! das ſich bey dir für alle Welt verbürgt,
    und das itzt an des Kreuzesſtamm
    der Schuldner Undank ſelbſt erwürgt.
    Wie ſeufzt das rechte Opferlamm,
    das nun des Vaters Zorn im Innerſten der Seele fühlt,
    das eine Angſt, die unbeſchreiblich iſt,
    aufs Grauſamſte durchwühlt.
    O Welt! ſieh! alle deine Sünden
    muß itzt das Lamm aufs Schrecklichſte empfinden!
    Von Gott verlaſſen! ach! wie unergründlich ſchwer
    quält dieſe Seelenangst! —

    \voice[Maria]
    \hspace*{3cm} Sohn! Heiligſter!
    War das die Seelenangſt, die in den finſtern Stunden,
    du Sündentilger! haſt empfunden?
    War darum jenes Licht, der Sonnenglanz, verſchwunden?

    \voice[Die Weiber und die Gläubigen]
    Angſt! Schrecklichſte! bey der die Sonne ſelbſt erblaßt!

    \voice[Lazarus]
    Wie unbeſchreiblich ſchwer preßt dich, Herr! unſrer Sünden Laſt!

    \voice[Eidli]
    Noch unbeſchreiblicher muß deine Marter ſeyn.

    \voice[Die Weiber und die Gläubigen]
    Verlaſſen! ach! von Gott biſt du, gerechtes Lamm!
    und doch, erretteſt du die undankbare Welt.

    \voice[Der römiſche Hauptmann]
    Wer dringt in dieß Geheimniß ein?
    Wie? daß ſogar ein Schaudern mich befällt?
  \end{movement}

  % \begin{movement}{}
  %   \voice[]
  % \end{movement}

  % \begin{movement}{}
  %   \voice[]
  % \end{movement}

  % \begin{movement}{}
  %   \voice[]
  % \end{movement}
}

\eesScore

\end{document}
